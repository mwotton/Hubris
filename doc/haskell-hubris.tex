% Local Variables:
% compile-command: "/opt/local/bin/pdflatex fp-syd.tex && open fp-syd.pdf"
% End:

\documentclass{beamer}
\usepackage{listings}
\usepackage{beamerthemesplit}

\title{Hubris}
\subtitle{A Trojan Horse for Haskell}
\author{Mark Wotton \textless mwotton@shimweasel.com\textgreater}
\date{\today}

\begin{document}
\lstset{language=Haskell}
\section{Two cultures}
\frame{\titlepage}

% \subsection
\begin{frame}
  \frametitle{I \ding{170} Ruby}
  \begin{itemize}
  \item concise and flexible
  \item Big web community, many libraries
  \item Fun
  \end{itemize}
\end{frame}

\begin{frame}
  \frametitle{I \ding{170} Haskell}
  \begin{itemize}
  \item<1-> Fast (optimised native code, multicore, etc)
  \item<2-> Expressive - type systems don't have to suck.
  \item<3-> Provably safe at compile time
  \end{itemize}
\end{frame}


\begin{frame}
  \frametitle{So, why do I care?}
  \begin{itemize}
  \item<1-> Make it easier to slap up a quick web interface to cool
    haskell code.
  \item<2-> Get more web-savvy devs into the Haskell community
  \end{itemize}
\end{frame}

\subsection{problems with Haskell}
\begin{frame}
\frametitle{heresies}
\setlength\parskip{0.1in}

Haskell web frameworks are still niche.

It's an engineering problem, and pretty boring.

Haskell devs aren't exactly hard to find, but they don't seem to be
web guys.

\end{frame}

\subsection{problems with Ruby}
\begin{frame}
  \frametitle{lies, damn lies, benchmarks}
  \center{JRuby vs GHC}
  \begin{tabular}{l l l l}
    Program &	        Time	&Memory &	Source Size\\ \hline
    reverse-complement	&5	&1	&1/4\\
    regex-dna	        &7	&3	&1/5\\
    binary-trees	        &8	&7	&1	\\
    k-nucleotide	        &10	&1	&1/7	\\
    pidigits	        &18	&18	&2	\\
    n-body	        &26	&53	&1	\\
    chameneos-redux	&30	&24	&1	\\
    fasta	                &31	&142	&1	\\
    fannkuch	        &45	&22	&1/4	\\
    spectral-norm	        &227	&56	&1/3	\\
    mandelbrot	        &319	&3	&1/2\\
  \end{tabular}
\end{frame}



\section{Hubris}
\begin{frame}
\frametitle{Peanut butter, meet chocolate}
Ruby has a heap of web frameworks, convenience libraries, well-tested
integration with javascript + CSS.

\setlength\parskip{0.25in}

Haskell is smoking fast with rock solid type safety but a relatively tiny
community

Hubris is my bridge between the two
\end{frame}

\begin{frame}
\frametitle{Again, WHY?}
There are seventy bazillion ways of talking between languages.
 \begin{itemize}
  \item Web services
  \item text over pipes
  \item Protocol buffers, Thrift
  \item COM, HOC, etc (binary interfaces)
 \end{itemize}

Reasons to do it this way:
 \begin{itemize}
   \item it's fun (for me, anyway)
   \item low fuss - no explicit mapping of function interfaces
   \item easy to explore a library
   \item few dependencies
 \end{itemize}
\end{frame}

\lstset{language=Haskell}
\subsection{Haskell example}
\begin{frame}[fragile]
  \frametitle{lazy, statically typed, and pure}
  Collatz.hs
  \begin{lstlisting}
module Collatz where
clMax lim = maximumBy (comparing snd) (assocs arr)
  where arr = listArray (1,lim)
                       (0:(map depth [2..]))
        step x = if even x
                   then div x 2
                   else 3 * x + 1
        depth x = 1 + if n <= lim
                        then arr ! n
                        else depth n
          where n = step x
  \end{lstlisting}
  Hubrify Collatz collatz.dylib Collatz.hs

\end{frame}

\subsection{wrap it in Ruby}
\begin{frame}[fragile]
  \frametitle{actually using it}
  \lstset{language=Ruby}
  \begin{lstlisting}
    require Hubris           # my favourite line
    module Collatz
      hubris :module => 'collatz'
    end
    puts Collatz.clMax(1000000)
    >> 837799
  \end{lstlisting}
\end{frame}

\subsection{beneath the covers}
\begin{frame}[fragile]
  \frametitle{on the Haskell side}
  Hubrify loads the given source files and attempts to find the
  passed-in module using the GHC API.
\lstset{language=Haskell}
\begin{lstlisting}
data RValue = T_NIL | T_FIXNUM Int | ...
type Value = CULong -- imported from Ruby
class Haskellable a where
  toHaskell :: RValue -> Maybe a

class Rubyable a where
  toRuby :: a -> RValue

wrap :: (Haskellable a, Rubyable b) ->
        (a -> b) -> (Value -> Value)
\end{lstlisting}
\end{frame}

\begin{frame}[fragile]
  \frametitle{blah}
  For each top level definition f, we then typecheck.
\begin{lstlisting}
(wrap f) T_NIL == T_NIL
\end{lstlisting}
if it fits, it's exportable.

  create a Haskell file exporting each valid function, add a top level
  manifest, and compile into a self-contained dylib.

\end{frame}

\begin{frame}
 \frametitle{On the ruby side}
 No Haskell-specific information at all.

 We have a manifest function we need to call to find the names of
 wrapped functions. (Also gives us a convenient point to initialise
 the Haskell runtime)

 attach all passed functions as methods on the surrounding module

\end{frame}


% \frametitle{Predictive}
\section{Making it less sucky}
\begin{frame}
\setlength\parskip{0.1in}
\frametitle{What's been done}
  caching of Haskell binaries

  transparent mapping - no boilerplate

  separation of concerns - no compiler on server

  ported to GHC
\end{frame}

\begin{frame}
\frametitle{Still to do}
  one-way bridge, no callbacks to Ruby

  autoconf support to find ruby libs and includes

  more serious hammering to find bugs - is it legitimate to dlopen
  multiple haskell dylibs?

  performance testing

\end{frame}


\begin{frame}
\frametitle{Try it out!}
  install GHC 6.12 release candidate

  git clone git://github.com/mwotton/HaskellHubris.git

  git clone git://github.com/mwotton/Hubris.git

  follow the README

  tell me what's missing

  patches very much welcome (thanks to Josh Price, James Britt and Tatsuhiro Ujihisa)
\end{frame}

\end{document}