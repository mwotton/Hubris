% Local Variables:
% compile-command: "/opt/local/bin/pdflatex fp-syd.tex && open fp-syd.pdf"
% End:

\documentclass{beamer}
\usepackage{listings}
\usepackage{beamerthemesplit}

\title{Hubris}
\subtitle{subjecting Haskell to Ruby's iron will}
\author{Mark Wotton \textless mwotton@shimweasel.com\textgreater}
\date{\today}

\begin{document}
\lstset{language=Haskell}
\section{Two cultures}
\frame{\titlepage}

% \subsection
\begin{frame}
  \frametitle{I \ding{170} Ruby}
  \begin{itemize}
  \item concise and flexible
  \item Big web community, many libraries
  \item Fun
  \end{itemize}
\end{frame}



\begin{frame}
  \frametitle{I \ding{170} Haskell}
  \begin{itemize}
  \item<1-> Fast (optimised native code, multicore, etc)
  \item<2-> Expressive - type systems don't have to suck.
  \item<3-> Provably safe at compile time
  \end{itemize}
\end{frame}


\subsection{lies, damn lies, benchmarks}
\begin{frame}
  \frametitle{the slide that's going to get me lynched}
  \center{JRuby vs GHC}
  \begin{tabular}{l l l l}
    Program &	        Time	&Memory &	Source Size\\ \hline
    reverse-complement	&5	&1	&1/4\\
    regex-dna	        &7	&3	&1/5\\
    binary-trees	        &8	&7	&1	\\
    k-nucleotide	        &10	&1	&1/7	\\
    pidigits	        &18	&18	&2	\\
    n-body	        &26	&53	&1	\\
    chameneos-redux	&30	&24	&1	\\
    fasta	                &31	&142	&1	\\
    fannkuch	        &45	&22	&1/4	\\
    spectral-norm	        &227	&56	&1/3	\\
    mandelbrot	        &319	&3	&1/2\\
  \end{tabular}
\end{frame}

\begin{frame}
  \frametitle{319 times? Really?}
  \begin{itemize}
  \item<1-> Haskell code written by expert hackers
  \item<2-> Longer in some cases - gone hard on optimisations
    even when they blow out source code size. But...
  \item <3-> 319 times faster, 1/3 the memory.
  \end{itemize}
\end{frame}

\subsection{problems with Haskell}
\begin{frame}
\frametitle{more lies to outrage Myles}
\setlength\parskip{0.1in}
there are approximately twelve programmers in the world who know Haskell

Nine are working on four different compilers with extensions for
solving logic puzzles in the type system (not kidding, google
``haskell instant insanity'')

The other three are working on six different web frameworks
\end{frame}

\section{Hubris}
\begin{frame}
\frametitle{Peanut butter, meet chocolate}
Ruby has a heap of web frameworks, convenience libraries, well-tested
integration with javascript + CSS
\setlength\parskip{0.25in}

Haskell is smoking fast with rock solid type safety but a tiny
community

Hubris is my bridge between the two
\end{frame}

\subsection{Haskell example}
\begin{frame}[fragile]
  \frametitle{lazy, statically typed, and pure}
  \begin{lstlisting}
clMax lim = maximumBy (comparing snd) (assocs arr)
  where arr = listArray (1,lim)
                       (0:(map depth [2..]))
        step x = if even x
                   then div x 2
                   else 3 * x + 1
        depth x = 1 + if n <= lim
                        then arr ! n
                        else depth n
          where n = step x
  \end{lstlisting}
\end{frame}


\subsection{wrap it in Ruby}
\begin{frame}[fragile]
  \frametitle{actually using it}
  \lstset{language=Ruby}
  \begin{lstlisting}
    require Hubris           # my favourite line
    c = Collatz.new          # any ruby object
    c.inline(haskell_string) # from above
    puts c.clMax(1000000)
    >> 837799
  \end{lstlisting}
\end{frame}

% \frametitle{Predictive}
\section{TODO, gotchas, FIXME}
\begin{frame}
\setlength\parskip{0.1in}
\frametitle{Making it less sucky}
Needs to use jhc, because ghc can't produce dynamic libs right now
(works with GHC HEAD, will be compatible with 6.12)

one-way bridge, no callbacks to Ruby

a smarter mapping layer

caching of Haskell binaries

cleanup of ruby interface

shore up support for arrays, hashes, BigInts

autoconf support to find ruby libs and includes

... lots to do
\end{frame}


\begin{frame}
\frametitle{Try it out!}
  install GHC and JHC

  git clone git://github.com/mwotton/Hubris.git

  follow the README

  tell me what's missing

  patches very much welcome (thanks to Josh Price, James Britt and Tatsuhiro Ujihisa)
\end{frame}


\begin{frame}
  \frametitle{Learning Haskell}
  Learn You A Haskell: why the lucky stiff's academic cousin

  Real World Haskell: awesome, practical introduction

  haskell channel on freenode

  beginners@haskell.org
\end{frame}
\end{document}